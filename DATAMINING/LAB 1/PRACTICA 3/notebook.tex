
% Default to the notebook output style

    


% Inherit from the specified cell style.




    
\documentclass[11pt]{article}

    
    
    \usepackage[T1]{fontenc}
    % Nicer default font (+ math font) than Computer Modern for most use cases
    \usepackage{mathpazo}

    % Basic figure setup, for now with no caption control since it's done
    % automatically by Pandoc (which extracts ![](path) syntax from Markdown).
    \usepackage{graphicx}
    % We will generate all images so they have a width \maxwidth. This means
    % that they will get their normal width if they fit onto the page, but
    % are scaled down if they would overflow the margins.
    \makeatletter
    \def\maxwidth{\ifdim\Gin@nat@width>\linewidth\linewidth
    \else\Gin@nat@width\fi}
    \makeatother
    \let\Oldincludegraphics\includegraphics
    % Set max figure width to be 80% of text width, for now hardcoded.
    \renewcommand{\includegraphics}[1]{\Oldincludegraphics[width=.8\maxwidth]{#1}}
    % Ensure that by default, figures have no caption (until we provide a
    % proper Figure object with a Caption API and a way to capture that
    % in the conversion process - todo).
    \usepackage{caption}
    \DeclareCaptionLabelFormat{nolabel}{}
    \captionsetup{labelformat=nolabel}

    \usepackage{adjustbox} % Used to constrain images to a maximum size 
    \usepackage{xcolor} % Allow colors to be defined
    \usepackage{enumerate} % Needed for markdown enumerations to work
    \usepackage{geometry} % Used to adjust the document margins
    \usepackage{amsmath} % Equations
    \usepackage{amssymb} % Equations
    \usepackage{textcomp} % defines textquotesingle
    % Hack from http://tex.stackexchange.com/a/47451/13684:
    \AtBeginDocument{%
        \def\PYZsq{\textquotesingle}% Upright quotes in Pygmentized code
    }
    \usepackage{upquote} % Upright quotes for verbatim code
    \usepackage{eurosym} % defines \euro
    \usepackage[mathletters]{ucs} % Extended unicode (utf-8) support
    \usepackage[utf8x]{inputenc} % Allow utf-8 characters in the tex document
    \usepackage{fancyvrb} % verbatim replacement that allows latex
    \usepackage{grffile} % extends the file name processing of package graphics 
                         % to support a larger range 
    % The hyperref package gives us a pdf with properly built
    % internal navigation ('pdf bookmarks' for the table of contents,
    % internal cross-reference links, web links for URLs, etc.)
    \usepackage{hyperref}
    \usepackage{longtable} % longtable support required by pandoc >1.10
    \usepackage{booktabs}  % table support for pandoc > 1.12.2
    \usepackage[inline]{enumitem} % IRkernel/repr support (it uses the enumerate* environment)
    \usepackage[normalem]{ulem} % ulem is needed to support strikethroughs (\sout)
                                % normalem makes italics be italics, not underlines
    

    
    
    % Colors for the hyperref package
    \definecolor{urlcolor}{rgb}{0,.145,.698}
    \definecolor{linkcolor}{rgb}{.71,0.21,0.01}
    \definecolor{citecolor}{rgb}{.12,.54,.11}

    % ANSI colors
    \definecolor{ansi-black}{HTML}{3E424D}
    \definecolor{ansi-black-intense}{HTML}{282C36}
    \definecolor{ansi-red}{HTML}{E75C58}
    \definecolor{ansi-red-intense}{HTML}{B22B31}
    \definecolor{ansi-green}{HTML}{00A250}
    \definecolor{ansi-green-intense}{HTML}{007427}
    \definecolor{ansi-yellow}{HTML}{DDB62B}
    \definecolor{ansi-yellow-intense}{HTML}{B27D12}
    \definecolor{ansi-blue}{HTML}{208FFB}
    \definecolor{ansi-blue-intense}{HTML}{0065CA}
    \definecolor{ansi-magenta}{HTML}{D160C4}
    \definecolor{ansi-magenta-intense}{HTML}{A03196}
    \definecolor{ansi-cyan}{HTML}{60C6C8}
    \definecolor{ansi-cyan-intense}{HTML}{258F8F}
    \definecolor{ansi-white}{HTML}{C5C1B4}
    \definecolor{ansi-white-intense}{HTML}{A1A6B2}

    % commands and environments needed by pandoc snippets
    % extracted from the output of `pandoc -s`
    \providecommand{\tightlist}{%
      \setlength{\itemsep}{0pt}\setlength{\parskip}{0pt}}
    \DefineVerbatimEnvironment{Highlighting}{Verbatim}{commandchars=\\\{\}}
    % Add ',fontsize=\small' for more characters per line
    \newenvironment{Shaded}{}{}
    \newcommand{\KeywordTok}[1]{\textcolor[rgb]{0.00,0.44,0.13}{\textbf{{#1}}}}
    \newcommand{\DataTypeTok}[1]{\textcolor[rgb]{0.56,0.13,0.00}{{#1}}}
    \newcommand{\DecValTok}[1]{\textcolor[rgb]{0.25,0.63,0.44}{{#1}}}
    \newcommand{\BaseNTok}[1]{\textcolor[rgb]{0.25,0.63,0.44}{{#1}}}
    \newcommand{\FloatTok}[1]{\textcolor[rgb]{0.25,0.63,0.44}{{#1}}}
    \newcommand{\CharTok}[1]{\textcolor[rgb]{0.25,0.44,0.63}{{#1}}}
    \newcommand{\StringTok}[1]{\textcolor[rgb]{0.25,0.44,0.63}{{#1}}}
    \newcommand{\CommentTok}[1]{\textcolor[rgb]{0.38,0.63,0.69}{\textit{{#1}}}}
    \newcommand{\OtherTok}[1]{\textcolor[rgb]{0.00,0.44,0.13}{{#1}}}
    \newcommand{\AlertTok}[1]{\textcolor[rgb]{1.00,0.00,0.00}{\textbf{{#1}}}}
    \newcommand{\FunctionTok}[1]{\textcolor[rgb]{0.02,0.16,0.49}{{#1}}}
    \newcommand{\RegionMarkerTok}[1]{{#1}}
    \newcommand{\ErrorTok}[1]{\textcolor[rgb]{1.00,0.00,0.00}{\textbf{{#1}}}}
    \newcommand{\NormalTok}[1]{{#1}}
    
    % Additional commands for more recent versions of Pandoc
    \newcommand{\ConstantTok}[1]{\textcolor[rgb]{0.53,0.00,0.00}{{#1}}}
    \newcommand{\SpecialCharTok}[1]{\textcolor[rgb]{0.25,0.44,0.63}{{#1}}}
    \newcommand{\VerbatimStringTok}[1]{\textcolor[rgb]{0.25,0.44,0.63}{{#1}}}
    \newcommand{\SpecialStringTok}[1]{\textcolor[rgb]{0.73,0.40,0.53}{{#1}}}
    \newcommand{\ImportTok}[1]{{#1}}
    \newcommand{\DocumentationTok}[1]{\textcolor[rgb]{0.73,0.13,0.13}{\textit{{#1}}}}
    \newcommand{\AnnotationTok}[1]{\textcolor[rgb]{0.38,0.63,0.69}{\textbf{\textit{{#1}}}}}
    \newcommand{\CommentVarTok}[1]{\textcolor[rgb]{0.38,0.63,0.69}{\textbf{\textit{{#1}}}}}
    \newcommand{\VariableTok}[1]{\textcolor[rgb]{0.10,0.09,0.49}{{#1}}}
    \newcommand{\ControlFlowTok}[1]{\textcolor[rgb]{0.00,0.44,0.13}{\textbf{{#1}}}}
    \newcommand{\OperatorTok}[1]{\textcolor[rgb]{0.40,0.40,0.40}{{#1}}}
    \newcommand{\BuiltInTok}[1]{{#1}}
    \newcommand{\ExtensionTok}[1]{{#1}}
    \newcommand{\PreprocessorTok}[1]{\textcolor[rgb]{0.74,0.48,0.00}{{#1}}}
    \newcommand{\AttributeTok}[1]{\textcolor[rgb]{0.49,0.56,0.16}{{#1}}}
    \newcommand{\InformationTok}[1]{\textcolor[rgb]{0.38,0.63,0.69}{\textbf{\textit{{#1}}}}}
    \newcommand{\WarningTok}[1]{\textcolor[rgb]{0.38,0.63,0.69}{\textbf{\textit{{#1}}}}}
    
    
    % Define a nice break command that doesn't care if a line doesn't already
    % exist.
    \def\br{\hspace*{\fill} \\* }
    % Math Jax compatability definitions
    \def\gt{>}
    \def\lt{<}
    % Document parameters
    \title{LAB01-PR03}
    
    
    

    % Pygments definitions
    
\makeatletter
\def\PY@reset{\let\PY@it=\relax \let\PY@bf=\relax%
    \let\PY@ul=\relax \let\PY@tc=\relax%
    \let\PY@bc=\relax \let\PY@ff=\relax}
\def\PY@tok#1{\csname PY@tok@#1\endcsname}
\def\PY@toks#1+{\ifx\relax#1\empty\else%
    \PY@tok{#1}\expandafter\PY@toks\fi}
\def\PY@do#1{\PY@bc{\PY@tc{\PY@ul{%
    \PY@it{\PY@bf{\PY@ff{#1}}}}}}}
\def\PY#1#2{\PY@reset\PY@toks#1+\relax+\PY@do{#2}}

\expandafter\def\csname PY@tok@w\endcsname{\def\PY@tc##1{\textcolor[rgb]{0.73,0.73,0.73}{##1}}}
\expandafter\def\csname PY@tok@c\endcsname{\let\PY@it=\textit\def\PY@tc##1{\textcolor[rgb]{0.25,0.50,0.50}{##1}}}
\expandafter\def\csname PY@tok@cp\endcsname{\def\PY@tc##1{\textcolor[rgb]{0.74,0.48,0.00}{##1}}}
\expandafter\def\csname PY@tok@k\endcsname{\let\PY@bf=\textbf\def\PY@tc##1{\textcolor[rgb]{0.00,0.50,0.00}{##1}}}
\expandafter\def\csname PY@tok@kp\endcsname{\def\PY@tc##1{\textcolor[rgb]{0.00,0.50,0.00}{##1}}}
\expandafter\def\csname PY@tok@kt\endcsname{\def\PY@tc##1{\textcolor[rgb]{0.69,0.00,0.25}{##1}}}
\expandafter\def\csname PY@tok@o\endcsname{\def\PY@tc##1{\textcolor[rgb]{0.40,0.40,0.40}{##1}}}
\expandafter\def\csname PY@tok@ow\endcsname{\let\PY@bf=\textbf\def\PY@tc##1{\textcolor[rgb]{0.67,0.13,1.00}{##1}}}
\expandafter\def\csname PY@tok@nb\endcsname{\def\PY@tc##1{\textcolor[rgb]{0.00,0.50,0.00}{##1}}}
\expandafter\def\csname PY@tok@nf\endcsname{\def\PY@tc##1{\textcolor[rgb]{0.00,0.00,1.00}{##1}}}
\expandafter\def\csname PY@tok@nc\endcsname{\let\PY@bf=\textbf\def\PY@tc##1{\textcolor[rgb]{0.00,0.00,1.00}{##1}}}
\expandafter\def\csname PY@tok@nn\endcsname{\let\PY@bf=\textbf\def\PY@tc##1{\textcolor[rgb]{0.00,0.00,1.00}{##1}}}
\expandafter\def\csname PY@tok@ne\endcsname{\let\PY@bf=\textbf\def\PY@tc##1{\textcolor[rgb]{0.82,0.25,0.23}{##1}}}
\expandafter\def\csname PY@tok@nv\endcsname{\def\PY@tc##1{\textcolor[rgb]{0.10,0.09,0.49}{##1}}}
\expandafter\def\csname PY@tok@no\endcsname{\def\PY@tc##1{\textcolor[rgb]{0.53,0.00,0.00}{##1}}}
\expandafter\def\csname PY@tok@nl\endcsname{\def\PY@tc##1{\textcolor[rgb]{0.63,0.63,0.00}{##1}}}
\expandafter\def\csname PY@tok@ni\endcsname{\let\PY@bf=\textbf\def\PY@tc##1{\textcolor[rgb]{0.60,0.60,0.60}{##1}}}
\expandafter\def\csname PY@tok@na\endcsname{\def\PY@tc##1{\textcolor[rgb]{0.49,0.56,0.16}{##1}}}
\expandafter\def\csname PY@tok@nt\endcsname{\let\PY@bf=\textbf\def\PY@tc##1{\textcolor[rgb]{0.00,0.50,0.00}{##1}}}
\expandafter\def\csname PY@tok@nd\endcsname{\def\PY@tc##1{\textcolor[rgb]{0.67,0.13,1.00}{##1}}}
\expandafter\def\csname PY@tok@s\endcsname{\def\PY@tc##1{\textcolor[rgb]{0.73,0.13,0.13}{##1}}}
\expandafter\def\csname PY@tok@sd\endcsname{\let\PY@it=\textit\def\PY@tc##1{\textcolor[rgb]{0.73,0.13,0.13}{##1}}}
\expandafter\def\csname PY@tok@si\endcsname{\let\PY@bf=\textbf\def\PY@tc##1{\textcolor[rgb]{0.73,0.40,0.53}{##1}}}
\expandafter\def\csname PY@tok@se\endcsname{\let\PY@bf=\textbf\def\PY@tc##1{\textcolor[rgb]{0.73,0.40,0.13}{##1}}}
\expandafter\def\csname PY@tok@sr\endcsname{\def\PY@tc##1{\textcolor[rgb]{0.73,0.40,0.53}{##1}}}
\expandafter\def\csname PY@tok@ss\endcsname{\def\PY@tc##1{\textcolor[rgb]{0.10,0.09,0.49}{##1}}}
\expandafter\def\csname PY@tok@sx\endcsname{\def\PY@tc##1{\textcolor[rgb]{0.00,0.50,0.00}{##1}}}
\expandafter\def\csname PY@tok@m\endcsname{\def\PY@tc##1{\textcolor[rgb]{0.40,0.40,0.40}{##1}}}
\expandafter\def\csname PY@tok@gh\endcsname{\let\PY@bf=\textbf\def\PY@tc##1{\textcolor[rgb]{0.00,0.00,0.50}{##1}}}
\expandafter\def\csname PY@tok@gu\endcsname{\let\PY@bf=\textbf\def\PY@tc##1{\textcolor[rgb]{0.50,0.00,0.50}{##1}}}
\expandafter\def\csname PY@tok@gd\endcsname{\def\PY@tc##1{\textcolor[rgb]{0.63,0.00,0.00}{##1}}}
\expandafter\def\csname PY@tok@gi\endcsname{\def\PY@tc##1{\textcolor[rgb]{0.00,0.63,0.00}{##1}}}
\expandafter\def\csname PY@tok@gr\endcsname{\def\PY@tc##1{\textcolor[rgb]{1.00,0.00,0.00}{##1}}}
\expandafter\def\csname PY@tok@ge\endcsname{\let\PY@it=\textit}
\expandafter\def\csname PY@tok@gs\endcsname{\let\PY@bf=\textbf}
\expandafter\def\csname PY@tok@gp\endcsname{\let\PY@bf=\textbf\def\PY@tc##1{\textcolor[rgb]{0.00,0.00,0.50}{##1}}}
\expandafter\def\csname PY@tok@go\endcsname{\def\PY@tc##1{\textcolor[rgb]{0.53,0.53,0.53}{##1}}}
\expandafter\def\csname PY@tok@gt\endcsname{\def\PY@tc##1{\textcolor[rgb]{0.00,0.27,0.87}{##1}}}
\expandafter\def\csname PY@tok@err\endcsname{\def\PY@bc##1{\setlength{\fboxsep}{0pt}\fcolorbox[rgb]{1.00,0.00,0.00}{1,1,1}{\strut ##1}}}
\expandafter\def\csname PY@tok@kc\endcsname{\let\PY@bf=\textbf\def\PY@tc##1{\textcolor[rgb]{0.00,0.50,0.00}{##1}}}
\expandafter\def\csname PY@tok@kd\endcsname{\let\PY@bf=\textbf\def\PY@tc##1{\textcolor[rgb]{0.00,0.50,0.00}{##1}}}
\expandafter\def\csname PY@tok@kn\endcsname{\let\PY@bf=\textbf\def\PY@tc##1{\textcolor[rgb]{0.00,0.50,0.00}{##1}}}
\expandafter\def\csname PY@tok@kr\endcsname{\let\PY@bf=\textbf\def\PY@tc##1{\textcolor[rgb]{0.00,0.50,0.00}{##1}}}
\expandafter\def\csname PY@tok@bp\endcsname{\def\PY@tc##1{\textcolor[rgb]{0.00,0.50,0.00}{##1}}}
\expandafter\def\csname PY@tok@fm\endcsname{\def\PY@tc##1{\textcolor[rgb]{0.00,0.00,1.00}{##1}}}
\expandafter\def\csname PY@tok@vc\endcsname{\def\PY@tc##1{\textcolor[rgb]{0.10,0.09,0.49}{##1}}}
\expandafter\def\csname PY@tok@vg\endcsname{\def\PY@tc##1{\textcolor[rgb]{0.10,0.09,0.49}{##1}}}
\expandafter\def\csname PY@tok@vi\endcsname{\def\PY@tc##1{\textcolor[rgb]{0.10,0.09,0.49}{##1}}}
\expandafter\def\csname PY@tok@vm\endcsname{\def\PY@tc##1{\textcolor[rgb]{0.10,0.09,0.49}{##1}}}
\expandafter\def\csname PY@tok@sa\endcsname{\def\PY@tc##1{\textcolor[rgb]{0.73,0.13,0.13}{##1}}}
\expandafter\def\csname PY@tok@sb\endcsname{\def\PY@tc##1{\textcolor[rgb]{0.73,0.13,0.13}{##1}}}
\expandafter\def\csname PY@tok@sc\endcsname{\def\PY@tc##1{\textcolor[rgb]{0.73,0.13,0.13}{##1}}}
\expandafter\def\csname PY@tok@dl\endcsname{\def\PY@tc##1{\textcolor[rgb]{0.73,0.13,0.13}{##1}}}
\expandafter\def\csname PY@tok@s2\endcsname{\def\PY@tc##1{\textcolor[rgb]{0.73,0.13,0.13}{##1}}}
\expandafter\def\csname PY@tok@sh\endcsname{\def\PY@tc##1{\textcolor[rgb]{0.73,0.13,0.13}{##1}}}
\expandafter\def\csname PY@tok@s1\endcsname{\def\PY@tc##1{\textcolor[rgb]{0.73,0.13,0.13}{##1}}}
\expandafter\def\csname PY@tok@mb\endcsname{\def\PY@tc##1{\textcolor[rgb]{0.40,0.40,0.40}{##1}}}
\expandafter\def\csname PY@tok@mf\endcsname{\def\PY@tc##1{\textcolor[rgb]{0.40,0.40,0.40}{##1}}}
\expandafter\def\csname PY@tok@mh\endcsname{\def\PY@tc##1{\textcolor[rgb]{0.40,0.40,0.40}{##1}}}
\expandafter\def\csname PY@tok@mi\endcsname{\def\PY@tc##1{\textcolor[rgb]{0.40,0.40,0.40}{##1}}}
\expandafter\def\csname PY@tok@il\endcsname{\def\PY@tc##1{\textcolor[rgb]{0.40,0.40,0.40}{##1}}}
\expandafter\def\csname PY@tok@mo\endcsname{\def\PY@tc##1{\textcolor[rgb]{0.40,0.40,0.40}{##1}}}
\expandafter\def\csname PY@tok@ch\endcsname{\let\PY@it=\textit\def\PY@tc##1{\textcolor[rgb]{0.25,0.50,0.50}{##1}}}
\expandafter\def\csname PY@tok@cm\endcsname{\let\PY@it=\textit\def\PY@tc##1{\textcolor[rgb]{0.25,0.50,0.50}{##1}}}
\expandafter\def\csname PY@tok@cpf\endcsname{\let\PY@it=\textit\def\PY@tc##1{\textcolor[rgb]{0.25,0.50,0.50}{##1}}}
\expandafter\def\csname PY@tok@c1\endcsname{\let\PY@it=\textit\def\PY@tc##1{\textcolor[rgb]{0.25,0.50,0.50}{##1}}}
\expandafter\def\csname PY@tok@cs\endcsname{\let\PY@it=\textit\def\PY@tc##1{\textcolor[rgb]{0.25,0.50,0.50}{##1}}}

\def\PYZbs{\char`\\}
\def\PYZus{\char`\_}
\def\PYZob{\char`\{}
\def\PYZcb{\char`\}}
\def\PYZca{\char`\^}
\def\PYZam{\char`\&}
\def\PYZlt{\char`\<}
\def\PYZgt{\char`\>}
\def\PYZsh{\char`\#}
\def\PYZpc{\char`\%}
\def\PYZdl{\char`\$}
\def\PYZhy{\char`\-}
\def\PYZsq{\char`\'}
\def\PYZdq{\char`\"}
\def\PYZti{\char`\~}
% for compatibility with earlier versions
\def\PYZat{@}
\def\PYZlb{[}
\def\PYZrb{]}
\makeatother


    % Exact colors from NB
    \definecolor{incolor}{rgb}{0.0, 0.0, 0.5}
    \definecolor{outcolor}{rgb}{0.545, 0.0, 0.0}



    
    % Prevent overflowing lines due to hard-to-break entities
    \sloppy 
    % Setup hyperref package
    \hypersetup{
      breaklinks=true,  % so long urls are correctly broken across lines
      colorlinks=true,
      urlcolor=urlcolor,
      linkcolor=linkcolor,
      citecolor=citecolor,
      }
    % Slightly bigger margins than the latex defaults
    
    \geometry{verbose,tmargin=1in,bmargin=1in,lmargin=1in,rmargin=1in}
    
    

    \begin{document}
    
    
    \maketitle
    
    

    
    \section{Laboratorio 1 - Práctica
3.}\label{laboratorio-1---pruxe1ctica-3.}

\subsubsection{Gonzalo de las Heras de Matías - Jorge de la Fuente
Tagarro}\label{gonzalo-de-las-heras-de-matuxedas---jorge-de-la-fuente-tagarro}

\subsection{Principal Component
Analysis.}\label{principal-component-analysis.}

\begin{figure}
\centering
\includegraphics{Images/usa.gif}
\caption{title}
\end{figure}

\subsubsection{Objetivo}\label{objetivo}

Existen casos en que las variables no se pueden representar visualmente
debido a que necesitaríamos varias dimensiones para ello. Para evitar
esto, existe una metodología la cual, un set de datos multidimensional,
podemos transformarlo para poder explicar gran parte de la información
en 2 o 3 dimensiones. Dicha metodología se conoce con el nombre de
Principal Component Analysis (PCA). Vamos a aplicarlo a un set de datos
que está colgado en Moodle y vamos a dar una serie de explicaciones de
que ocurre.

El dataset contiene es un archivo csv con distintos atributos para cada
atentado registrado.

    \paragraph{Librerias}\label{librerias}

    \begin{Verbatim}[commandchars=\\\{\}]
{\color{incolor}In [{\color{incolor}6}]:} \PY{k+kn}{import} \PY{n+nn}{pandas} \PY{k}{as} \PY{n+nn}{pd}
        \PY{k+kn}{import} \PY{n+nn}{matplotlib}\PY{n+nn}{.}\PY{n+nn}{pyplot} \PY{k}{as} \PY{n+nn}{plt}
        \PY{k+kn}{import} \PY{n+nn}{numpy} \PY{k}{as} \PY{n+nn}{np}
        \PY{k+kn}{from} \PY{n+nn}{matplotlib}\PY{n+nn}{.}\PY{n+nn}{pyplot} \PY{k}{import} \PY{n}{cm}
        \PY{k+kn}{from} \PY{n+nn}{sklearn}\PY{n+nn}{.}\PY{n+nn}{preprocessing} \PY{k}{import} \PY{n}{StandardScaler}
        \PY{k+kn}{from} \PY{n+nn}{sklearn}\PY{n+nn}{.}\PY{n+nn}{decomposition} \PY{k}{import} \PY{n}{PCA}
\end{Verbatim}


    \subsubsection{Ejercicio 1}\label{ejercicio-1}

    \paragraph{Lo primero que habrá que hacer será estandarizar los datos
para que las diferencias de rango no supongan un problema a la hora de
procesar la información. Usa para ello el método StandardScaler de la
librería
scikitk-learn.}\label{lo-primero-que-habruxe1-que-hacer-seruxe1-estandarizar-los-datos-para-que-las-diferencias-de-rango-no-supongan-un-problema-a-la-hora-de-procesar-la-informaciuxf3n.-usa-para-ello-el-muxe9todo-standardscaler-de-la-libreruxeda-scikitk-learn.}

    Cargamos el fichero de datos, eliminamos la columna referente a los
paises para quedarnos solo con los valores numéricos y centramos los
valores.

    \begin{Verbatim}[commandchars=\\\{\}]
{\color{incolor}In [{\color{incolor}7}]:} \PY{c+c1}{\PYZsh{} Carga del fichero de datos.}
        \PY{n}{Dataset} \PY{o}{=} \PY{n}{pd}\PY{o}{.}\PY{n}{read\PYZus{}excel}\PY{p}{(}\PY{l+s+s2}{\PYZdq{}}\PY{l+s+s2}{Datos/PCA.xlsx}\PY{l+s+s2}{\PYZdq{}}\PY{p}{,} \PY{n}{header}\PY{o}{=}\PY{l+m+mi}{0}\PY{p}{)} 
        
        \PY{c+c1}{\PYZsh{} Guardado de la columna de estados.}
        \PY{n}{Estados} \PY{o}{=} \PY{n}{pd}\PY{o}{.}\PY{n}{DataFrame}\PY{p}{(}\PY{p}{)}
        \PY{n}{Estados}\PY{p}{[}\PY{l+s+s2}{\PYZdq{}}\PY{l+s+s2}{nombre}\PY{l+s+s2}{\PYZdq{}}\PY{p}{]} \PY{o}{=} \PY{n}{Dataset}\PY{p}{[}\PY{l+s+s2}{\PYZdq{}}\PY{l+s+s2}{State}\PY{l+s+s2}{\PYZdq{}}\PY{p}{]}
        
        \PY{c+c1}{\PYZsh{} Borrado de la columna de estados.}
        \PY{k}{del} \PY{n}{Dataset}\PY{p}{[}\PY{l+s+s1}{\PYZsq{}}\PY{l+s+s1}{State}\PY{l+s+s1}{\PYZsq{}}\PY{p}{]}
        
        \PY{c+c1}{\PYZsh{} Centralización de los datos.}
        \PY{n}{datos\PYZus{}centrados} \PY{o}{=} \PY{n}{StandardScaler}\PY{p}{(}\PY{p}{)}\PY{o}{.}\PY{n}{fit\PYZus{}transform}\PY{p}{(}\PY{n}{Dataset}\PY{o}{.}\PY{n}{values}\PY{p}{)}
\end{Verbatim}


    \subsubsection{Ejercicio 2}\label{ejercicio-2}

    \paragraph{El segundo paso será a partir de los datos anteriores,
obtener los autovalores (eigenvalues) y los autovectores (eigenvectors)
que nos permitan explicar cuantos componentes necesitamos para
representar los datos iniciales. Para ello primer habrá que obtener la
matriz de covarianza mediante el método cov de numpy y después aplicarle
a dicha matriz el método linalg.eig también de numpy. Obten un DataFrame
con el porcentaje de varianza y el acumulado por cada componente.
Explica que quieren decir estos
datos.}\label{el-segundo-paso-seruxe1-a-partir-de-los-datos-anteriores-obtener-los-autovalores-eigenvalues-y-los-autovectores-eigenvectors-que-nos-permitan-explicar-cuantos-componentes-necesitamos-para-representar-los-datos-iniciales.-para-ello-primer-habruxe1-que-obtener-la-matriz-de-covarianza-mediante-el-muxe9todo-cov-de-numpy-y-despuuxe9s-aplicarle-a-dicha-matriz-el-muxe9todo-linalg.eig-tambiuxe9n-de-numpy.-obten-un-dataframe-con-el-porcentaje-de-varianza-y-el-acumulado-por-cada-componente.-explica-que-quieren-decir-estos-datos.}

    Capturamos las componentes principales, los autovalores y autovectores.

    \begin{Verbatim}[commandchars=\\\{\}]
{\color{incolor}In [{\color{incolor}8}]:} \PY{c+c1}{\PYZsh{} Usamos la libreria de sklearn par proyectar los datos .}
        \PY{n}{modelo\PYZus{}PCA} \PY{o}{=} \PY{n}{PCA}\PY{p}{(}\PY{n}{n\PYZus{}components}\PY{o}{=}\PY{l+m+mi}{6}\PY{p}{)}
        
        \PY{c+c1}{\PYZsh{} Capturamos las componentes principales.}
        \PY{n}{componentes\PYZus{}principales} \PY{o}{=} \PY{n}{modelo\PYZus{}PCA}\PY{o}{.}\PY{n}{fit\PYZus{}transform}\PY{p}{(}\PY{n}{datos\PYZus{}centrados}\PY{p}{)}
\end{Verbatim}


    Creamos el dataframe con los autovalores y el \% de varianza y el \%
acumulado.

    \begin{Verbatim}[commandchars=\\\{\}]
{\color{incolor}In [{\color{incolor}9}]:} \PY{c+c1}{\PYZsh{} Creamos el dataframe.}
        \PY{n}{AutoValores} \PY{o}{=} \PY{n}{pd}\PY{o}{.}\PY{n}{DataFrame}\PY{p}{(}\PY{p}{)}
        
        \PY{c+c1}{\PYZsh{} Confeccionamos el dataframe con las columnas deseadas.}
        \PY{n}{AutoValores}\PY{p}{[}\PY{l+s+s2}{\PYZdq{}}\PY{l+s+s2}{componente}\PY{l+s+s2}{\PYZdq{}}\PY{p}{]} \PY{o}{=} \PY{p}{[}\PY{l+s+s2}{\PYZdq{}}\PY{l+s+s2}{comp\PYZus{}1}\PY{l+s+s2}{\PYZdq{}}\PY{p}{,} \PY{l+s+s2}{\PYZdq{}}\PY{l+s+s2}{comp\PYZus{}2}\PY{l+s+s2}{\PYZdq{}}\PY{p}{,} \PY{l+s+s2}{\PYZdq{}}\PY{l+s+s2}{comp\PYZus{}3}\PY{l+s+s2}{\PYZdq{}}\PY{p}{,} \PY{l+s+s2}{\PYZdq{}}\PY{l+s+s2}{comp\PYZus{}4}\PY{l+s+s2}{\PYZdq{}}\PY{p}{,} \PY{l+s+s2}{\PYZdq{}}\PY{l+s+s2}{comp\PYZus{}5}\PY{l+s+s2}{\PYZdq{}}\PY{p}{,} \PY{l+s+s2}{\PYZdq{}}\PY{l+s+s2}{comp\PYZus{}6}\PY{l+s+s2}{\PYZdq{}}\PY{p}{]}
        \PY{n}{AutoValores}\PY{p}{[}\PY{l+s+s2}{\PYZdq{}}\PY{l+s+s2}{autovalor}\PY{l+s+s2}{\PYZdq{}}\PY{p}{]} \PY{o}{=} \PY{n}{modelo\PYZus{}PCA}\PY{o}{.}\PY{n}{explained\PYZus{}variance\PYZus{}}
        \PY{n}{AutoValores}\PY{p}{[}\PY{l+s+s2}{\PYZdq{}}\PY{l+s+s2}{\PYZpc{}}\PY{l+s+s2}{ varianza}\PY{l+s+s2}{\PYZdq{}}\PY{p}{]} \PY{o}{=} \PY{n}{modelo\PYZus{}PCA}\PY{o}{.}\PY{n}{explained\PYZus{}variance\PYZus{}ratio\PYZus{}} \PY{o}{*} \PY{l+m+mi}{100}
        \PY{n}{AutoValores}\PY{p}{[}\PY{l+s+s2}{\PYZdq{}}\PY{l+s+s2}{\PYZpc{}}\PY{l+s+s2}{ acum}\PY{l+s+s2}{\PYZdq{}}\PY{p}{]} \PY{o}{=} \PY{n}{modelo\PYZus{}PCA}\PY{o}{.}\PY{n}{explained\PYZus{}variance\PYZus{}ratio\PYZus{}}\PY{o}{.}\PY{n}{cumsum}\PY{p}{(}\PY{p}{)} \PY{o}{*} \PY{l+m+mi}{100}
        
        \PY{c+c1}{\PYZsh{} Mostramos el dataframe.}
        \PY{n}{AutoValores}
\end{Verbatim}


\begin{Verbatim}[commandchars=\\\{\}]
{\color{outcolor}Out[{\color{outcolor}9}]:}   componente  autovalor  \% varianza      \% acum
        0     comp\_1   2.274435   37.163966   37.163966
        1     comp\_2   1.393388   22.767771   59.931737
        2     comp\_3   1.111548   18.162548   78.094285
        3     comp\_4   0.626345   10.234397   88.328682
        4     comp\_5   0.394775    6.450576   94.779258
        5     comp\_6   0.319509    5.220742  100.000000
\end{Verbatim}
            
    Observamos que:

\begin{verbatim}
    <li>Con la comp_1 representamos el 37.1% del patrón.</li>
    <li>Con la comp_2 representamos el 22.7% del patrón, junto a la anterior componente representamos el 59.9% del patrón</li>
    <li>Con la comp_3 representamos el 18.1% del patrón, junto a las anteriores componentes representamos el 78.0% del patrón.</li>
    <li>Con la comp_4 representamos el 10.2% del patrón, junto a las anteriores componentes representamos el 88.3% del patrón.</li>
    <li>Con la comp_4 representamos el  6.4% del patrón, junto a las anteriores componentes representamos el 94.7% del patrón.</li>
    <li>Con la comp_5 representamos el  5.2% del patrón, junto a las anteriores componentes representamos el 100.% del patrón.</li>
</ul>
\end{verbatim}

    \subsubsection{Ejercicio 3}\label{ejercicio-3}

    \paragraph{Por último queremos representar gráficamente los individuos
de nuestro dataset pero usando los valores de los componentes
principales obtenidas. Obtén un diagrama de dispersión en 2 dimensiones
y comenta que has interpretado en
él.}\label{por-uxfaltimo-queremos-representar-gruxe1ficamente-los-individuos-de-nuestro-dataset-pero-usando-los-valores-de-los-componentes-principales-obtenidas.-obtuxe9n-un-diagrama-de-dispersiuxf3n-en-2-dimensiones-y-comenta-que-has-interpretado-en-uxe9l.}

    Creamos el dataframe con las distintas componentes.

    \begin{Verbatim}[commandchars=\\\{\}]
{\color{incolor}In [{\color{incolor}10}]:} \PY{c+c1}{\PYZsh{} Creamos el dataframe con los autovalores.}
         \PY{n}{Componentes} \PY{o}{=} \PY{n}{pd}\PY{o}{.}\PY{n}{DataFrame}\PY{p}{(}\PY{n}{modelo\PYZus{}PCA}\PY{o}{.}\PY{n}{components\PYZus{}} \PY{p}{,}\PY{n}{columns}\PY{o}{=}\PY{p}{[}\PY{l+s+s2}{\PYZdq{}}\PY{l+s+s2}{comp\PYZus{}1}\PY{l+s+s2}{\PYZdq{}}\PY{p}{,} \PY{l+s+s2}{\PYZdq{}}\PY{l+s+s2}{comp\PYZus{}2}\PY{l+s+s2}{\PYZdq{}}\PY{p}{,} \PY{l+s+s2}{\PYZdq{}}\PY{l+s+s2}{comp\PYZus{}3}\PY{l+s+s2}{\PYZdq{}}\PY{p}{,} \PY{l+s+s2}{\PYZdq{}}\PY{l+s+s2}{comp\PYZus{}4}\PY{l+s+s2}{\PYZdq{}}\PY{p}{,} \PY{l+s+s2}{\PYZdq{}}\PY{l+s+s2}{comp\PYZus{}5}\PY{l+s+s2}{\PYZdq{}}\PY{p}{,} \PY{l+s+s2}{\PYZdq{}}\PY{l+s+s2}{comp\PYZus{}6}\PY{l+s+s2}{\PYZdq{}}\PY{p}{]}\PY{p}{)}
         
         \PY{c+c1}{\PYZsh{} Añadimos la característica.}
         \PY{n}{Componentes}\PY{p}{[}\PY{l+s+s2}{\PYZdq{}}\PY{l+s+s2}{variable}\PY{l+s+s2}{\PYZdq{}}\PY{p}{]} \PY{o}{=} \PY{n}{Dataset}\PY{o}{.}\PY{n}{keys}\PY{p}{(}\PY{p}{)}
         
         \PY{c+c1}{\PYZsh{} Mostramos el dataframe.}
         \PY{n}{Componentes}
\end{Verbatim}


\begin{Verbatim}[commandchars=\\\{\}]
{\color{outcolor}Out[{\color{outcolor}10}]:}      comp\_1    comp\_2    comp\_3    comp\_4    comp\_5    comp\_6  \textbackslash{}
         0  0.326413  0.145422 -0.326732  0.506363  0.471164 -0.535868   
         1 -0.632376 -0.093329 -0.570093 -0.397401  0.212780 -0.251359   
         2 -0.265517  0.896224  0.078631  0.052648 -0.294214 -0.175402   
         3 -0.132939  0.110223  0.648810 -0.298119  0.662657 -0.145722   
         4 -0.141069 -0.362440  0.340322  0.049018 -0.440271 -0.732582   
         5 -0.620872 -0.153011  0.159010  0.701156  0.114953  0.246957   
         
                               variable  
         0             Total Population  
         1       Net Domestic Migration  
         2    Civilian move from abroad  
         3  Net International Migration  
         4                Period Births  
         5                Period Deaths  
\end{Verbatim}
            
    Para representar el dataset en 2 dimensiones solo necesitamos las dos
primeras componentes.

    La componente comp\_1 indica:

\begin{verbatim}
    <li>Las variables 'Period Deaths' y 'Net Domestic Migration' al tener ser en valor absoluto la mayor cantidad, son las características que más infliyen en esta componente.</li>
\end{verbatim}

Por el contrario, 'Period Births' y 'Net International Migration' son
las que menos influyen al ser los menores valores en valor absoluto.

\begin{verbatim}
    <li>Las variables que más influyen al ser negativas influyen cuanto menos cantidad tengan.</li>
</ul>
\end{verbatim}

    La componente comp\_2 indica:

\begin{verbatim}
    <li>La variable 'Civilian move from abroad' al tener ser en valor absoluto la mayor cantidad, es la característica que más infliye en esta componente.</li>
\end{verbatim}

Por el contrario, 'Net Domestic Migration' y 'Net International
Migration' son las que menos influyen al ser los menores valores en
valor absoluto.

\begin{verbatim}
    <li>La variable que más influye al ser positiva influye cuanto más cantidad tenga.</li>
</ul>
\end{verbatim}

    \begin{Verbatim}[commandchars=\\\{\}]
{\color{incolor}In [{\color{incolor}12}]:} \PY{c+c1}{\PYZsh{} Creamos el dataframe con las componentes principales calculadas para cada registro.}
         \PY{n}{Componentes\PYZus{}Principales} \PY{o}{=} \PY{n}{pd}\PY{o}{.}\PY{n}{DataFrame}\PY{p}{(}\PY{n}{data}\PY{o}{=}\PY{n}{componentes\PYZus{}principales}\PY{p}{,} \PY{n}{columns}\PY{o}{=}\PY{p}{[}\PY{l+s+s2}{\PYZdq{}}\PY{l+s+s2}{comp\PYZus{}1}\PY{l+s+s2}{\PYZdq{}}\PY{p}{,} \PY{l+s+s2}{\PYZdq{}}\PY{l+s+s2}{comp\PYZus{}2}\PY{l+s+s2}{\PYZdq{}}\PY{p}{,} \PY{l+s+s2}{\PYZdq{}}\PY{l+s+s2}{comp\PYZus{}3}\PY{l+s+s2}{\PYZdq{}}\PY{p}{,}
                                                                                      \PY{l+s+s2}{\PYZdq{}}\PY{l+s+s2}{comp\PYZus{}4}\PY{l+s+s2}{\PYZdq{}}\PY{p}{,} \PY{l+s+s2}{\PYZdq{}}\PY{l+s+s2}{comp\PYZus{}5}\PY{l+s+s2}{\PYZdq{}}\PY{p}{,} \PY{l+s+s2}{\PYZdq{}}\PY{l+s+s2}{comp\PYZus{}6}\PY{l+s+s2}{\PYZdq{}}\PY{p}{]}\PY{p}{)}
         \PY{c+c1}{\PYZsh{} Añadimos la columna con el nombre del estado.}
         \PY{n}{Componentes\PYZus{}Principales}\PY{p}{[}\PY{l+s+s2}{\PYZdq{}}\PY{l+s+s2}{estado}\PY{l+s+s2}{\PYZdq{}}\PY{p}{]} \PY{o}{=} \PY{n}{Estados}\PY{p}{[}\PY{l+s+s2}{\PYZdq{}}\PY{l+s+s2}{nombre}\PY{l+s+s2}{\PYZdq{}}\PY{p}{]}
         
         \PY{c+c1}{\PYZsh{} Mostramos el dataframe.}
         \PY{n}{Componentes\PYZus{}Principales}\PY{o}{.}\PY{n}{head}\PY{p}{(}\PY{l+m+mi}{10}\PY{p}{)}
\end{Verbatim}


\begin{Verbatim}[commandchars=\\\{\}]
{\color{outcolor}Out[{\color{outcolor}12}]:}      comp\_1    comp\_2    comp\_3    comp\_4    comp\_5    comp\_6  \textbackslash{}
         0 -1.189427  0.078246 -0.526890  0.602713 -0.736593 -0.150520   
         1  2.909501  3.663137 -0.173503 -0.931222  0.636790 -0.787041   
         2  1.609022 -0.246804  1.692072  0.984550 -0.612221  0.373891   
         3 -1.299644 -0.003577 -0.164837  0.684678 -0.845151  0.187450   
         4  4.099445 -3.532106 -1.379766 -0.677760  0.307037 -1.092108   
         5  1.669286  0.487463  1.438153  0.341347  0.604915 -0.141485   
         6 -0.516061 -0.404299  0.005988 -0.424923  0.557454  0.663026   
         7 -0.254278  0.480146  0.839449  0.337760 -0.222955  0.240183   
         8 -0.071416 -0.094366 -1.121775 -0.882105 -0.742052  2.197604   
         9  0.718970 -2.510599  1.302755 -1.052498 -1.135604  0.053199   
         
                          estado  
         0               Alabama  
         1                Alaska  
         2               Arizona  
         3              Arkansas  
         4            California  
         5              Colorado  
         6           Connecticut  
         7              Delaware  
         8  District of Columbia  
         9               Florida  
\end{Verbatim}
            
    Ahora gracias a las PCA hemos conseguido proyectar los datos en menos
dimensiones podemos emplear 2 para representarlo en 2 dimensiones y que
gracias al procentaje de varianza acumulado anteriormente, sabemos que
representamos los patrones con casi el 60\% de completitud.

    \begin{Verbatim}[commandchars=\\\{\}]
{\color{incolor}In [{\color{incolor}16}]:} \PY{c+c1}{\PYZsh{} Preparamos la gráfica.}
         \PY{n}{fig}\PY{p}{,} \PY{n}{ax} \PY{o}{=} \PY{n}{plt}\PY{o}{.}\PY{n}{subplots}\PY{p}{(}\PY{n}{figsize}\PY{o}{=}\PY{p}{(}\PY{l+m+mi}{20}\PY{p}{,} \PY{l+m+mi}{10}\PY{p}{)}\PY{p}{)}
         \PY{n}{plt}\PY{o}{.}\PY{n}{scatter}\PY{p}{(}\PY{n}{Componentes\PYZus{}Principales}\PY{p}{[}\PY{l+s+s2}{\PYZdq{}}\PY{l+s+s2}{comp\PYZus{}1}\PY{l+s+s2}{\PYZdq{}}\PY{p}{]}\PY{p}{,} \PY{n}{Componentes\PYZus{}Principales}\PY{p}{[}\PY{l+s+s2}{\PYZdq{}}\PY{l+s+s2}{comp\PYZus{}2}\PY{l+s+s2}{\PYZdq{}}\PY{p}{]}\PY{p}{,} \PY{n}{c}\PY{o}{=}\PY{l+s+s2}{\PYZdq{}}\PY{l+s+s2}{b}\PY{l+s+s2}{\PYZdq{}}\PY{p}{)}
         \PY{k}{for} \PY{n}{i}\PY{p}{,} \PY{n}{txt} \PY{o+ow}{in} \PY{n+nb}{enumerate}\PY{p}{(}\PY{n}{Componentes\PYZus{}Principales}\PY{p}{[}\PY{l+s+s2}{\PYZdq{}}\PY{l+s+s2}{estado}\PY{l+s+s2}{\PYZdq{}}\PY{p}{]}\PY{p}{)}\PY{p}{:}
             \PY{n}{pos} \PY{o}{=} \PY{p}{(}\PY{n}{Componentes\PYZus{}Principales}\PY{p}{[}\PY{l+s+s2}{\PYZdq{}}\PY{l+s+s2}{comp\PYZus{}1}\PY{l+s+s2}{\PYZdq{}}\PY{p}{]}\PY{p}{[}\PY{n}{i}\PY{p}{]} \PY{o}{+} \PY{l+m+mf}{0.05}\PY{p}{,} \PY{n}{Componentes\PYZus{}Principales}\PY{p}{[}\PY{l+s+s2}{\PYZdq{}}\PY{l+s+s2}{comp\PYZus{}2}\PY{l+s+s2}{\PYZdq{}}\PY{p}{]}\PY{p}{[}\PY{n}{i}\PY{p}{]}\PY{p}{)}
             \PY{n}{ax}\PY{o}{.}\PY{n}{annotate}\PY{p}{(}\PY{n}{txt}\PY{p}{,} \PY{p}{(}\PY{n}{Componentes\PYZus{}Principales}\PY{p}{[}\PY{l+s+s2}{\PYZdq{}}\PY{l+s+s2}{comp\PYZus{}1}\PY{l+s+s2}{\PYZdq{}}\PY{p}{]}\PY{p}{[}\PY{n}{i}\PY{p}{]}\PY{p}{,} \PY{n}{Componentes\PYZus{}Principales}\PY{p}{[}\PY{l+s+s2}{\PYZdq{}}\PY{l+s+s2}{comp\PYZus{}2}\PY{l+s+s2}{\PYZdq{}}\PY{p}{]}\PY{p}{[}\PY{n}{i}\PY{p}{]}\PY{p}{)}\PY{p}{,} \PY{n}{pos}\PY{p}{,} \PY{l+s+s1}{\PYZsq{}}\PY{l+s+s1}{data}\PY{l+s+s1}{\PYZsq{}}\PY{p}{,} \PYZbs{}
                         \PY{n}{size}\PY{o}{=}\PY{l+m+mi}{8}\PY{p}{)}
         \PY{n}{plt}\PY{o}{.}\PY{n}{grid}\PY{p}{(}\PY{k+kc}{True}\PY{p}{)}
         \PY{n}{plt}\PY{o}{.}\PY{n}{xlabel}\PY{p}{(}\PY{l+s+s1}{\PYZsq{}}\PY{l+s+s1}{comp\PYZus{}1}\PY{l+s+s1}{\PYZsq{}}\PY{p}{)}
         \PY{n}{plt}\PY{o}{.}\PY{n}{ylabel}\PY{p}{(}\PY{l+s+s1}{\PYZsq{}}\PY{l+s+s1}{comp\PYZus{}2}\PY{l+s+s1}{\PYZsq{}}\PY{p}{)}
         \PY{n}{plt}\PY{o}{.}\PY{n}{title}\PY{p}{(}\PY{l+s+s1}{\PYZsq{}}\PY{l+s+s1}{Datos proyectados en 2 dimensiones}\PY{l+s+s1}{\PYZsq{}}\PY{p}{)}
         \PY{n}{plt}\PY{o}{.}\PY{n}{show}\PY{p}{(}\PY{p}{)}
\end{Verbatim}


    \begin{center}
    \adjustimage{max size={0.9\linewidth}{0.9\paperheight}}{output_23_0.png}
    \end{center}
    { \hspace*{\fill} \\}
    
    En el eje X tenemos la primera componente 'comp\_1' y en el eje Y la
segunda componente 'comp\_2'. En el diagrama se juntarán aquellos
patrones con Period Deaths' y 'Net Domestic Migration' bajos (son
negativos y los más influyentes en 'comp\_1') y con más cantidad de
'Civilian move from abroad' (es positivo y el más influyente en
'comp\_2')


    % Add a bibliography block to the postdoc
    
    
    
    \end{document}
